\documentclass{beamer}
\usecolortheme{owl}
%\usetheme{Boadilla}

%\theoremstyle{definition}
%\newtheorem{definition}{Definition}[section]


\begin{document}

\section{1. Rounding Error}
\begin{frame}{1.1 Floating-Point Formats: Definition}
  A total of four parameters are associated with floating-point number representations:
  $p, \beta, e_{\text{min}}, e_{\text{max}}$

  $$
    \pm d_{0}.d_{1}d_{2} \cdots d_{p-1} \times \beta^{e} = \pm\left( d_{0} + d_{1}\beta^{-1} + \cdots + d_{p-1}\beta^{-(p-1)} \right) \beta^{e},
  $$
  where
  $$
  \begin{aligned}
    & 0 \le d_{t} < \beta \quad\forall\; t = 0, 1, 2, \cdots, p-1, \\
    & e_{\text{min}} \le e \le e_{\text{max}}.
  \end{aligned}
  $$
  There are therefore \textbf{at most} $\beta^{p} \cdot (e_{\text{max}} - e_{\text{min}} + 1) \cdot 2$ floating-point numbers.
  ("\textbf{at most}" because the representation is not unique, e.g. $0.01 \times 10^{1} = 1.00 \times 10^{-1}.$
  The $2$ due to $\pm$ signs.)
  This means that one can encode them all using
  $$
    \lg\left( {\beta^{p} \cdot (e_{\text{max}} - e_{\text{min}} + 1) \cdot 2} \right) =
    \lg(\beta^{p}) + \lg( e_{\text{max}} - e_{\text{min}} + 1) + 1
  $$
  bits.

  %\vspace{10pt}
  %\textbf{Rmk.}
  %\begin{itemize}
  %  \item<2-> The subspace $V$ is independent of the choice of $x_{0} \in C.$
  %  \item<3-> We call any subset $C \subseteq E$ satisfying the above property an affine subset.
  %\end{itemize}
\end{frame}


\begin{frame}{1.1 Floating-Point Formats: Q\&A}
  \textbf{(?1)} Why $0.1_{10} = 1.1001100110011 \times 2^{-4}$ ($p=14, \beta=2$), i.e. why the repeating pattern of $0011$?

  \vspace{7pt}
\end{frame}


\begin{frame}{1.1 Floating-Point Formats: Normalization}
  We briefly mentioned that so far the floating-point representation is not unique.
  For example, $0.01 \times 10^{1} = 1.00 \times 10^{-1}$ and zero can be represented
  by $0.00 \times 10^{e}$ for any $e.$ (Sous-entendu $\beta = 10, p = 3.$)

  We could make floating-point representation unique by requiring additionally that $d_{0} \gneq 0.$
\end{frame}


\section{2. IEEE Standard}
\begin{frame}{2.1.5 Cones}
  %%\begin{definition}(Cone)
  %\begin{definition}
  %  A set $C$ is said to be \textit{nonnegative homogeneous} or a \textit{cone} if $\forall x \in C, \theta \ge 0$, we
  %  have $\theta x \in C$.
  %\end{definition}
\end{frame}


\end{document}
