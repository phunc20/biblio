\documentclass{beamer}
\usecolortheme{owl}
%\usetheme{Boadilla}

%\theoremstyle{definition}
%\newtheorem{definition}{Definition}[section]


\begin{document}

\section{2.1.2 Affine sets}
%\subsection{sub a}
\begin{frame}{2.1.2 Affine sets}
  \textbf{(?)} Does $\theta_{1} x_{1} + \theta_{2} x_{2} + \theta_{3} x_{3}$, where $x_{1}, x_{2}, x_{3} \in \mathbb{R}^{n}$
  and $\theta_{1}, \theta_{2}, \theta_{3} \in \mathbb{R}$ satisfy
  $$
    \theta_{1} + \theta_{2} + \theta_{3} = 1\,,
  $$
  constitute a plane in $\mathbb{R}^{n}$ (i.e. the plane containing all three points of $x_{1}, x_{2}, x_{3}$)?
\end{frame}


\section{2.1.5 Cones}
\begin{frame}{2.1.5 Cones}
  %\begin{definition}(Cone)
  \begin{definition}
    A set $C$ is said to be \textit{nonnegative homogeneous} or a \textit{cone} if $\forall x \in C, \theta \ge 0$, we
    have $\theta x \in C$.
  \end{definition}

  In particular, a cone must \textit{contain the origin}.

    \textbf{(?)} A set $C$ is a convex cone \textbf{iff} $\forall x_{1}, x_{2} \in C, \theta_{1}, \theta_{2} \ge 0$, we have
    $\theta_{1}x_{1} + \theta_{2}x_{2} \in C$.\\
    \textbf{(R)} Assume that $\theta_{1} + \theta_{2} \gneq 0$, we have
    $
      \frac{\theta_{1}}{\theta_{1} + \theta_{2}} x_{1}
      + \frac{\theta_{2}}{\theta_{1} + \theta_{2}} x_{2} \in C
    $ because $C$ is convex. Then we can multiply this vector by $\theta_{1} + \theta_{2}$ and
    it will still remain inside $C$ because $C$ is a cone, whence $\theta_{1}x_{1} + \theta_{2}x_{2} \in C$
\end{frame}


\end{document}
