\documentclass{beamer}
\usecolortheme{owl}
%\usetheme{Boadilla}

%\theoremstyle{definition}
%\newtheorem{definition}{Definition}[section]


\begin{document}

\section{2.1.2 Affine sets}
%\subsection{sub a}
\begin{frame}{2.1.2 Affine Sets: Definition}
  Let $E$ be a vector space over $\mathbb{R}$.
  \begin{theorem}
    Let $C$ be a subset of $E$. We have the following:\\
    \vspace{5pt}
    $C = V + x_{0}$, where $V$ is a subspace of $E$ and $x_{0} \in C \iff \forall y_{1}, y_{2}
    \in C, \theta_{1}, \theta_{2} \in \mathbb{R}$ s.t. $\theta_{1} + \theta_{2} = 1,$ we have
    $\theta_{1} y_{1} + \theta_{2} y_{2} \in C.$
  \end{theorem}

  \vspace{10pt}
  \textbf{Rmk.}
  \begin{itemize}
    \item<2-> The subspace $V$ is independent of the choice of $x_{0} \in C.$
    \item<3-> We call any subset $C \subseteq E$ satisfying the above property an affine subset.
  \end{itemize}
\end{frame}


\begin{frame}{2.1.2 Affine sets: Affine Combination}
  \textbf{(?)} Does $\theta_{1} x_{1} + \theta_{2} x_{2} + \theta_{3} x_{3}$, where $x_{1}, x_{2}, x_{3} \in \mathbb{R}^{n}$
  and $\theta_{1}, \theta_{2}, \theta_{3} \in \mathbb{R}$ satisfy
  $$
    \theta_{1} + \theta_{2} + \theta_{3} = 1\,,
  $$
  constitute a plane in $\mathbb{R}^{n}$ (i.e. the plane containing all three points of $x_{1}, x_{2}, x_{3}$)?

  \vspace{10pt}
  \textbf{hull} in English has a meaning of \textit{The outer covering of a fruit or seed}, synonymous to \textit{shell}.
  I think this is the origin of naming \textit{affine/convex hull}, etc.
\end{frame}


\section{2.1.5 Cones}
\begin{frame}{2.1.5 Cones}
  %\begin{definition}(Cone)
  \begin{definition}
    A set $C$ is said to be \textit{nonnegative homogeneous} or a \textit{cone} if $\forall x \in C, \theta \ge 0$, we
    have $\theta x \in C$.
  \end{definition}

  \vspace{7pt}
  In particular, a cone must \textit{contain the origin}.

  %\textbf{(?)} A set $C$ is a convex cone \textbf{iff} $\forall x_{1}, x_{2} \in C, \theta_{1}, \theta_{2} \ge 0$, we have
  %$\theta_{1}x_{1} + \theta_{2}x_{2} \in C$.\\
  %\textbf{(R)} Assume that $\theta_{1} + \theta_{2} \gneq 0$, we have
  %$
  %  \frac{\theta_{1}}{\theta_{1} + \theta_{2}} x_{1}
  %  + \frac{\theta_{2}}{\theta_{1} + \theta_{2}} x_{2} \in C
  %$ because $C$ is convex. Then we can multiply this vector by $\theta_{1} + \theta_{2}$ and
  %it will still remain inside $C$ because $C$ is a cone, whence $\theta_{1}x_{1} + \theta_{2}x_{2} \in C$
\end{frame}


\begin{frame}{2.1.5 Cones: Q\&A}
  \textbf{(?1)} Does there exist any \textbf{nonconvex} cone?\\
  \textbf{(R1)} Yes. For example, in $\mathbb{R}^{2}$, let $C$ be the cone of two rays
  $$
    C = \left\{k\begin{pmatrix} 1\\1 \end{pmatrix}\;|\; k \ge 0 \right\}
        \cup \left\{k\begin{pmatrix} -1\\1 \end{pmatrix}\;|\; k \ge 0 \right\}
  $$
  (TODO: Add a TikZ visualization here!)
  Then clearly $C$ is not convex.

  We could also see that a ray seems to be the most basic (nonempty) cone (component).
\end{frame}


\begin{frame}{2.1.5 Cones: Q\&A}
  \textbf{(?2)} A set $C$ is a convex cone \textbf{iff} $\forall x_{1}, x_{2} \in C, \theta_{1}, \theta_{2} \ge 0$, we have
  $\theta_{1}x_{1} + \theta_{2}x_{2} \in C$.\\
  \textbf{(R2)} Assume that $\theta_{1} + \theta_{2} \gneq 0$, we have
  $
    \frac{\theta_{1}}{\theta_{1} + \theta_{2}} x_{1}
    + \frac{\theta_{2}}{\theta_{1} + \theta_{2}} x_{2} \in C
  $ because $C$ is convex. Then we can multiply this vector by $\theta_{1} + \theta_{2}$ and
  it will still remain inside $C$ because $C$ is a cone, whence $\theta_{1}x_{1} + \theta_{2}x_{2} \in C.$
  Note that this looks almost like the definition of a subspace.
\end{frame}

\end{document}
